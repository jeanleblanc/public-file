\documentclass[12pt]{article}
\usepackage[margin=1in]{geometry} 
\usepackage{fancyhdr} 
\usepackage{lipsum}  
\usepackage{hyperref}
\usepackage{xcolor}


% Title
\pagestyle{fancy}
\fancyhf{} 
\renewcommand{\headrulewidth}{0pt} \fancyhead[C]{\today \hfill  \LARGE \textbf{Animation Road Map}  \hfill \normalsize Jean A. Aboutboul} 

% Command references
\newcommand{\myreference}[2]{
  \item \textcolor{black}{#1} \ifx&#2&\else \href{#2}{\textcolor{blue}{\textit{#2}}}\fi
}


\begin{document}

Ressources from \href{https://www.reddit.com/r/animation/}{Reddit animation}

L'animation, c'est beaucoup plus de la pratique que de la théorie, mais il y
a quand même un peu de théorie. En général, les "cours" commencent par faire
animer une balle qui accélère et décélère.\\
Plusieurs types d'animation : Traditional 2D, digital 2D, 3D, stop motion. \\

\section*{Théorie}
\href{https://en.wikipedia.org/wiki/Twelve_basic_principles_of_animation#Squash_and_stretch}{12 principes de l'animation Wikipedia}\\
\href{https://www.youtube.com/watch?v=uDqjIdI4bF4}{12 principes de l'animation série vidéo}\\
\href{https://www.amazon.com/Animators-Survival-Kit-Richard-Williams/dp/0571202284}{Animator's survival kit by Richard Williams}\\
\href{https://www.youtube.com/watch?v=MPRAH7AKMWw}{Composition (YouTube channel)}\\
\section*{Pratique}
\href{http://www.angryanimator.com/word/2010/11/26/animation-tutorial-1-bouncing-ball/:w}{In depth guide of animating a bouncing ball}\\
\href{https://www.youtube.com/watch?v=B0J27sf9N1Y&list=PLjEaoINr3zgEPv5y--4MKpciLaoQYZB1Z&pp=iAQB donut}{3D donut with Blender}\\
\href{https://thinkinganimation.com/animation-exercises/#1505506796027-3-4}{Série d'exercices progressifs}\\
\href{https://www.animatorisland.com/51-great-animation-exercises-to-master/}{Autre série d'exercices}\\
\href{https://www.khanacademy.org/computing/pixar}{Cours Pixar}\\
Tu peux aussi aller sur une vidéo YouTube et utiliser les touches \texttt{,} et \texttt{.} pour
voir image par image et recopier.\\
À noter : essaye d'avoir des critiques constructives extérieures, utilise des références,
sauvegarde tout ce que tu fais, apprends aussi à animer un cycle de marche.\\
Conseils : Essaye stop motion, évite le rotoscoping, va voir les techniques de \textbf{Tex Avery},
observe les mouvements exagérés dans la vie, redessine les poses floues (pour mieux comprendre les mouvements
rapides entre les images), varie les expressions et micro-expressions, regarde la simplicité des
\textbf{puppets de Jim Henson}, les expressions minimalistes de \textbf{Chuck Jones}.

\section*{Dessin}
Un peu évident, mais pour bien animer il faut aussi bien dessiner. Je trouve ce livre super,
il a un livre d'exercices associé.
\href{https://www.amazon.com/New-Drawing-Right-Side-Brain/dp/0874774241/ref=sr_1_4?crid=2J16E9R86P4RS&dib=eyJ2IjoiMSJ9.Xuhg0WHBK64jwG3xzdJak41d2NUI1Ym7pF-5jjmkWqUf6LTXIY93Rj40w_EqEfqVCU7Qs5TNvqavsLll7HtHbiAASWLoXv8iPXLZxRRP9TYzF1Q59hOrpPMT7BpCNjQVwrGrk5SmJXtRNQQYw3--q_dJ3WjzZnJsF9T7OfDCIeT_PHFnBFBrCgalNvoEq_n4r8V6vy4jwemPA5vwmlrbieq-gQmI7UgyxJUqaDgzrmA.kl4TePk8hQsJYhxWBS9mySUZZ2rgduyNIVGgJ9NUngE&dib_tag=se&keywords=Drawing+on+the+Right+Side+of+the+Brain&qid=1713625911&s=books&sprefix=drawing+on+the+right+side+of+the+brain%2Cstripbooks-intl-ship%2C157&sr=1-4}{Drawing on the right side of the brain}

\section*{Guide de temps}
Comme toute compétence pratique, le plus important est de pratiquer. Cependant, c'est un peu
difficile de pratiquer sans théorie, donc au début il est plus important de se concentrer en majorité sur la théorie
et petit à petit de faire de plus en plus de pratique. Également, comme
pour toute compétence pratique, la constance a un impact remarquable, mieux vaut 10 fois 6 minutes qu'une fois 60 minutes.
Alors moi je te propose comme ça, après tu en fais ce que tu veux : \\
Par jour (minimum, faire plus est permis)
\begin{itemize}
\item 10 minutes d'exercices d'animation (d'abord les deux premiers, puis les deux séries en même temps, vu que les exercices se ressemblent un peu,
c'est une bonne idée de les faire 2 fois, genre tu fais 1 série 1 puis 1 série 2 et ainsi de suite)
\item 20 minutes de théorie (commence par regarder la vidéo puis le Wikipedia, puis le livre)
\item 15 minutes de théorie de dessin (lis le livre)
\item 5 minutes de pratique de dessin (dessine juste ce que tu vois, le plus vite possible et surtout sans gomme)
\end{itemize}

Une fois que tu as fini la théorie, tu peux mettre le temps pour la théorie en pratique (10 + 20 et 15 + 5)\\
C'est important pour la théorie d'avoir une bonne méthode d'apprentissage, tu peux regarder la chaîne de
\href{https://www.youtube.com/@ArtemKirsanov}{ce mec} ou au moins \href{https://youtu.be/RVB3PBPxMWg?si=pxJQ9ZSSEITsK11r}{cette vidéo}
si tu veux avoir une compréhension technique et en profondeur de ça, mais sinon retiens juste : \\
Prends des notes À LA MAIN (surtout si c'est une vidéo), le même jour, remets tes notes au propre. Plus tard (par exemple
à la fin d'un chapitre), essaye de faire un \href{https://youtu.be/qic-4OTM9yI?si=h4JxZlchLxyfWI2p}{mind map} (ou \href{https://youtu.be/8ueBiBOTs9E?si=IJpvLEDKzlWHy2ql}{ici}) du chapitre (ou jusqu'où tu es arrivé).
\section*{Logiciels}
\subsection*{Pour iPad}
Le plus recommandé : \href{https://www.roughanimator.com/}{RoughAnimator}\\
Deuxième plus recommandé : \href{https://apps.apple.com/us/app/toonsquid/id1573778812}{ToonSquid}\\
Autre (moins conseillé) : \href{https://procreate.com/dreams}{Procreate Dreams}\\
Plus cher : \href{https://callipeg.com}{Callipeg}\\
\subsubsection*{Pour PC}
Krita \& OpenToonz (animation 2D image par image)\\
Blender (animation 3D, 2D image par image)\\
After Effects (motion design)\\
Toon Boom (animation 2D avec rigging)\\
\subsection*{Médias}
\begin{itemize}
    \item \href{https://www.youtube.com/c/ArtSideofLife/featured}{Art Side of Life}
    \item \href{https://www.youtube.com/c/TyCarterArt/videos}{Ty Carter Art}
    \item \href{https://www.youtube.com/c/lulusketches}{Lulu Sketches}
    \item \href{https://www.youtube.com/c/bluefley00}{Marc Brunet}
    \item \href{https://www.youtube.com/c/Rossdraws}{Ross Draws}
    \item \href{https://www.youtube.com/user/sinixdesign/featured}{Sinix Design}
    \item \href{https://www.youtube.com/c/JelArts}{Jel Arts}
    \item \href{https://www.youtube.com/c/AhmedAldoori}{Ahmed Aldoori}
    \item \href{https://www.youtube.com/c/sakimichanOfficalArt}{Sakimichan}
    \item \href{https://www.youtube.com/c/DrawingTutorialsOnline}{Drawing Tutorials Online}
    \item \href{https://www.youtube.com/c/ArtofWeiHo}{Art of Wei Ho}
    \item \href{https://www.youtube.com/c/Toonboxstudio}{Toon Box Studio}
    \item \href{https://www.youtube.com/c/CGMasterAcademyCGMA}{CG Master Academy}
    \item \href{https://www.youtube.com/c/ProkoTV}{Proko}
    \item \href{https://www.youtube.com/c/reiqws}{Reiq}
    \item \href{https://www.youtube.com/c/ChrisLegaspi_art}{Chris Legaspi}
    \item \href{https://www.youtube.com/channel/UC1UD10HC9qZo1qfdl-OV9kw}{Lois van Baarle}
    \item \href{https://www.youtube.com/c/StephenSilver}{Stephen Silver}
    \item \href{https://www.youtube.com/c/DaveGreco}{Dave Greco}
    \item \href{https://www.youtube.com/c/TrentKaniuga}{Trent Kaniuga}
    \item \href{https://www.youtube.com/user/AnimatorIslandTV}{Animator Island}
    \item \href{https://www.youtube.com/user/0oBlademastero0/featured}{Blade Master}
    \item \href{https://www.youtube.com/user/gnomonschool/featured}{Gnomon}
    \item \href{https://www.youtube.com/c/DoKiDoKiDrawing/featured}{DoKiDoKi Drawing}
    \item \href{https://www.youtube.com/c/DaoLeTrong/featured}{Dao Le Trong}
    \item \href{https://www.youtube.com/user/marcobucci/featured}{Marco Bucci}
    \item \href{https://www.youtube.com/channel/UCmc-Zy6PxGcaVq8OlYVjX8A/featured}{Kienan Lafferty}
    \item \href{https://www.youtube.com/user/skiptothelove/featured}{Schoolism}
    \item \href{https://www.youtube.com/c/thevirtualinstructor}{The Virtual Instructor}
    \item \href{https://www.youtube.com/c/mikeymegamega}{Mikey Mega Mega}
    \item \href{https://www.youtube.com/c/AMBAnimationAcademy}{AMB Animation Academy}
    \item \href{https://www.youtube.com/c/HowardWimshurst}{Howard Wimshurst}
    \item \href{https://youtube.com/c/VictorStaris}{Victor Staris}
\end{itemize}

\end{document}
