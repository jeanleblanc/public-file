Depuis mon plus jeune âge, j'ai toujours été passionné par les sciences et l'apprentissage. Cette curiosité 
innée m'a naturellement orienté vers des études d'ingénieur, avec la volonté de contribuer à un monde 
meilleur grâce à mes compétences et ma créativité.\\
Actuellement étudiant en génie mécanique à l'EPFL, j'ai découvert un intérêt particulier pour la 
robotique. C'est un domaine porteur d'avenir qui offre de nombreuses possibilités d'applications dans des 
secteurs variés, tels que l'écologie, la médecine ou encore l'industrie. La robotique a le potentiel d'apporter 
des solutions innovantes et efficaces à des problèmes concrets.\\
Au-delà de ses applications, la robotique me séduit par son caractère multidisciplinaire. Elle permet de combiner différents champs 
passionnants, comme les neurosciences, l'intelligence artificielle ou même l'aérospatial. C'est un domaine qui offre un large éventail de 
possibilités pour un esprit curieux.\\
Mais ce qui me motive réellement, c'est la perspective de pouvoir concrétiser mes propres projets. Grâce aux connaissances et aux compétences 
que j'acquiers, je serai en mesure de donner vie à mes idées, de manière autonome ou en collaboration.
C'est un défi stimulant et gratifiant de mener à bien son projet de A à Z.\\
Parallèlement à mes compétences techniques, j'ai développé de solides aptitudes en gestion de projet et en leadership d'équipe. À travers les différents projets que j'ai menés, tant académiques que personnels, j'ai démontré ma capacité à concevoir des plans de conceptualisation clairs et structurés, à coordonner efficacement le travail d'une équipe et à maintenir une communication fluide entre les membres. J'apprécie particulièrement le processus de création collaboratif et je sais m'adapter aux différents profils et 
compétences de chacun pour tirer le meilleur du groupe. 
Ces expériences m'ont appris à être organisé, à anticiper les défis et 
à proposer des solutions créatives pour atteindre les objectifs 
fixés dans les délais impartis.\\
Parallèlement à mes études, je suis un grand passionné de musique. Je joue du piano depuis l'âge de trois ans et demi, 
explorant principalement le répertoire classique, mais également le jazz. La musique est un élément central de ma vie.\\
En résumé, je suis un étudiant passionné par les sciences et la robotique, désireux de mettre mes compétences et ma créativité au 
service de projets porteurs de sens. Avec détermination et enthousiasme, je suis prêt à relever les défis qui m'attendent pour atteindre mes 
objectifs et contribuer positivement à mon échelle.

