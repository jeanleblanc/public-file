\documentclass{article}
\begin{document}
\section{Information personelles}
JEAN ABOUTBOUL\\
Adresse (Suisse) : 14 chemin de la forêt, 1024 Ecublens, Vaud, Suisse\\
Adresse (Belgique) : 59 montagne de St Job, 1180 Uccle, Bruxelles, Belgique\\
Téléphone (Suisse) : +41 78 259 25 97\\
WhatsApp (Belgique) : +32 470 666 043\\
E-mail personnel : jeanaboutboul@icloud.com, jeanaboutboul@gmail.com\\
E-mail universitaire : jean.aboutboul@epfl.ch\\
Date de naissance : 26/09/2005\\
\textbf{Langues}  :
\begin{itemize}
\item Français : langue maternelle
\item Anglais : niveau C1
\item Néerlandais : bases (éducation bruxelloise)
\end{itemize}
\section{Résumé}
Étudiant en première année de génie mécanique à l'EPFL, passionné par la robotique et la microtechnique. 
Solides compétences en programmation (Python, C, Bash), conception mécanique (CATIA, Fusion 360) et gestion de projet.
Expérience pratique à travers le projet CanSat (conception et lancement d'un nano-satellite) et un stage chez Ajinomotrix (solutions d'IA pour l'industrie).
Recherche un stage stimulant en robotique pour contribuer à des projets innovants et approfondir mes connaissances dans ce domaine passionnant.
\section{A propos de moi}
Depuis mon plus jeune âge, j'ai toujours été passionné par les sciences et l'apprentissage. Cette curiosité 
innée m'a naturellement orienté vers des études d'ingénieur, avec la volonté de contribuer à un monde 
meilleur grâce à mes compétences et ma créativité.\\
Actuellement étudiant en génie mécanique à l'EPFL, j'ai découvert un intérêt particulier pour la 
robotique. C'est un domaine porteur d'avenir qui offre de nombreuses possibilités d'applications dans des 
secteurs variés, tels que l'écologie, la médecine ou encore l'industrie. La robotique a le potentiel d'apporter 
des solutions innovantes et efficaces à des problèmes concrets.\\
Au-delà de ses applications, la robotique me séduit par son caractère multidisciplinaire. Elle permet de combiner différents champs 
passionnants, comme les neurosciences, l'intelligence artificielle ou même l'aérospatial. C'est un domaine qui offre un large éventail de 
possibilités pour un esprit curieux.\\
Mais ce qui me motive réellement, c'est la perspective de pouvoir concrétiser mes propres projets. Grâce aux connaissances et aux compétences 
que j'acquiers, je serai en mesure de donner vie à mes idées, de manière autonome ou en collaboration.
C'est un défi stimulant et gratifiant de mener à bien son projet de A à Z.\\
Parallèlement à mes compétences techniques, j'ai développé de solides aptitudes en gestion de projet et en leadership d'équipe. À travers les différents projets que j'ai menés, tant académiques que personnels, j'ai démontré ma capacité à concevoir des plans de conceptualisation clairs et structurés, à coordonner efficacement le travail d'une équipe et à maintenir une communication fluide entre les membres. J'apprécie particulièrement le processus de création collaboratif et je sais m'adapter aux différents profils et 
compétences de chacun pour tirer le meilleur du groupe. 
Ces expériences m'ont appris à être organisé, à anticiper les défis et 
à proposer des solutions créatives pour atteindre les objectifs 
fixés dans les délais impartis.\\
Parallèlement à mes études, je suis un grand passionné de musique. Je joue du piano depuis l'âge de trois ans et demi, 
explorant principalement le répertoire classique, mais également le jazz. La musique est un élément central de ma vie.\\
En résumé, je suis un étudiant passionné par les sciences et la robotique, désireux de mettre mes compétences et ma créativité au 
service de projets porteurs de sens. Avec détermination et enthousiasme, je suis prêt à relever les défis qui m'attendent pour atteindre mes 
objectifs et contribuer positivement à mon échelle.


\section{FORMATION}
\textbf{École Polytechnique Fédérale de Lausanne (EPFL), Suisse}\\
Bachelor en Génie Mécanique, 1ère année\\
Cours suivis :
\begin{itemize}
\item Ba1 : Analyse I, Algèbre linéaire, Physique mécanique, Chimie des matériaux, Science et technologie de l'électricité, Construction mécanique
\item Ba2 : Analysis 2 (en anglais), Physique thermodynamique, Mechanical structure (en anglais), Informatique (théorie et programmation), Enjeux mondiaux : Communication (autour de l'IA), Projet de construction mécanique
\end{itemize}

\textbf{École Decroly, Bruxelles, Belgique}\\
Diplôme d'Enseignement Secondaire Supérieur (CESS), Orientation Mathématiques (6h) et Sciences (7h)\\
Moyenne générale : 80,4\%\\
Résultats notables :
\begin{itemize}
\item Mathématiques (6h) : 92,5\%
\item Physique (3h) : 88,5\%
\item Biologie (2h) : 90\%
\item Philosophie renforcée (2h, cours supplémentaire) : 82,5\%
\end{itemize}

Diplôme du Premier Degré de l'Enseignement Secondaire (CE1D)\\
Moyenne générale : 86\%\\
Résultats notables : Mathématiques (93\%), Sciences (88\%), Français (82\%), Néerlandais (79\%)\\

\textbf{École Hamaide, Bruxelles, Belgique}\\
Diplôme du Certificat d'Études de Base (CEB)\\
Moyenne générale : 95\%\\
Résultats notables :
\begin{itemize}
\item Mathématiques (grandeurs, nombres et opérations, solides et figures) : 100\%
\item Éveil (initiation scientifique) : 96\%
\item Français (écouter, écrire, lire fictionnel) : 93\%
\end{itemize}


\section{EXPÉRIENCES PROFESSIONNELLES}
\textbf{Stage chez Ajinomotrix, Bruxelles}
\textit{2 semaines}
\begin{itemize}
\item Stage auprès de François Wayenberg, PDG et fondateur d'Ajinomatrix, une entreprise pionnière dans le développement de logiciels de mesure sensorielle utilisant l'intelligence artificielle
\item Découverte des activités de l'entreprise, qui propose des solutions innovantes pour interpréter les mesures sensorielles du goût et de l'odorat à l'aide de l'IA, en collaboration avec des consommateurs, des panels de juges de dégustation et des capteurs tels que des nez et des bouches électroniques
\item Ajinomatrix, fondée dans les années 2000, s'est développée à partir d'une application spécifique pour devenir une plateforme open source offrant une large gamme de services visant à simplifier la vie des utilisateurs et à révolutionner l'industrie sensorielle
\item Le fondateur, François Wayenberg, possède une expérience de plusieurs années dans la R\&D alimentaire au sein d'une grande entreprise japonaise, ainsi qu'une expertise dans des projets de pointe en collaboration avec la NASA et des projets expérimentaux avancés en Israël. Il est également auteur, poète, cinéaste, inventeur et entrepreneur en série.
\end{itemize}

\textbf{Pianiste accompagnateur}
\begin{itemize}
\item Accompagnement musical lors de divers événements et cérémonies
\end{itemize}

\textbf{Chef d'un mouvement de jeunesse, JJL, Bruxelles}
\textit{2 ans}
\begin{itemize}
\item Encadrement et animation d'activités pour les jeunes
\item Développement de compétences en leadership et en gestion d'équipe
\end{itemize}

\textbf{Travail en restaurant étoilé "Canne en Ville", Bruxelles}
\begin{itemize}
\item Expérience en restauration haut de gamme
\item Développement de compétences en service clientèle et en travail d'équipe dans un environnement exigeant
\end{itemize}




\section{COMPÉTENCES TECHNIQUES}
\textbf{Logiciels de conception et de modélisation}\\
CATIA : Compétences apprise première année d'université\\
Fusion 360 : compétences de base\\
Blender : compétences de base\\

\textbf{Programmation et développement}\\
\LaTeX : niveau très avancé (appris sur internet, notamment avec les tutoriels Overleaf et des vidéos YouTube)\\
Langages de programmation : Bash, Python, Julia, C, HTML, CSS, JavaScript, JSON\\
Théorie de l'informatique : connaissances acquises via le cours "Crash Course Computer Science" sur YouTube, des cours universitaires de l'EPFL et de Berkeley en ligne, et le livre "Introduction à la théorie de la computation" de Michael Sipser\\
Linux : maîtrise du système d'exploitation (4 ans d'expérience utilisateur, connaissances approfondies grâce au livre "How Linux Works" de Brian Ward)\\
Base de données : compétences de base en SQL\\
Outils de développement : Git, GitHub/GitLab, Markdown, Vim, Neovim, Manim\\

\textbf{Intelligence Artificielle}\\
Machine Learning : compétences de base (références principales : spécialisation "Machine Learning" sur Coursera par Stanford et Andrew Ng, cours de 3Blue1Brown sur YouTube)\\

\textbf{Musique}\\
Piano : compétences avancées\\
Théorie musicale : compétences avancées (références principales : musictheory.net et "Jazz Theory" de Mark Levine)\\
Logiciel de production musicale : compétences de base sur Logic Pro\\

\textbf{Graphisme et créativité}\\

\textbf{CONNAISSANCES SCIENTIFIQUES COMPLÉMENTAIRES}\\
\textbf{Mathématiques}\\
\textbf{Physique}\\
\textbf{Neurosciences et biologie}

Je tiens à souligner que je ne considère aucune de mes compétences comme définitivement acquise. 
Je suis conscient qu'il y a toujours de la place pour l'amélioration et l'approfondissement des connaissances.\\
De plus, comme mentionné dans la section "À propos de moi", 
j'ai une véritable passion pour l'apprentissage. Je suis donc 
tout à fait disposé et enthousiaste à l'idée d'acquérir de 
nouvelles compétences ou d'améliorer celles que je possède déjà, 
dans n'importe quel domaine qui serait nécessaire ou souhaité pour mener à bien les projets 
qui me seront confiés.\\

\section{Projets}
\subsection{Projet Académique}
\textbf{Conception d'un treuil de forage mécanique (CATIA)}
\begin{itemize}
\item Conception complète d'un treuil de forage mécanique sur CATIA, de la modélisation 3D aux calculs physiques
\item Compétences développées : conception mécanique, modélisation 3D, analyse de forces et de contraintes
\end{itemize}

\textbf{Création d'un poster sur une problématique liée à l'intelligence artificielle}
\begin{itemize}
\item Réalisation d'un poster scientifique présentant une problématique spécifique dans le domaine de l'IA
\item Compétences développées : synthèse d'informations, communication scientifique, vulgarisation
\end{itemize}

\textbf{Olympiades nationales de physique}
\begin{itemize}
\item Participation aux Olympiades nationales de physique, un concours mettant en avant les connaissances et la réflexion scientifique
\item Compétences développées : résolution de problèmes physiques complexes, raisonnement scientifique, gestion du stress
\end{itemize}

\textbf{Projet CanSat}
\begin{itemize}
\item Participation au projet CanSat, qui consiste à concevoir, construire et lancer un nano-satellite de la taille d'une canette
\item Compétences développées : travail d'équipe, gestion de projet, conception de systèmes embarqués, analyse de données
\end{itemize}

\textbf{Création d'un jeu de type "Candy Crush" en langage C}
\begin{itemize}
\item Développement d'un jeu inspiré de Candy Crush en utilisant le langage de programmation C
\item Compétences développées : programmation en C, conception d'algorithmes, gestion de la logique de jeu
\end{itemize}

\subsection{PROJETS PERSONNELS}
\textbf{JeanFlix : Plateforme personnelle de streaming vidéo}
\begin{itemize}
\item Création d'une plateforme de streaming vidéo personnelle, incluant le développement du site web, l'hébergement et la gestion d'une base de données
\item Compétences développées : développement web (HTML, CSS, JavaScript), gestion de base de données, hébergement web
\end{itemize}

\textbf{Site web CV (portfolio)}
\begin{itemize}
\item Création d'un site web vitrine dynamique servant de portfolio en ligne
\item Compétences développées : développement web (HTML, CSS, JavaScript), design web, hébergement
\end{itemize}

\textbf{Site web sur la musique classique et l'analyse de pièces}
\begin{itemize}
\item Développement d'un site web dédié à la musique classique, permettant de publier des analyses de pièces sans avoir à refaire le site
\item Compétences développées : développement web, rédaction de contenu, analyse musicale, vulgarisation
\end{itemize}

\textbf{Création d'une intelligence artificielle de recommandation musicale}
\begin{itemize}
\item Développement d'un système de recommandation musicale basé sur l'intelligence artificielle
\item Compétences développées : traitement de données, algorithmes de recommandation, apprentissage automatique
\end{itemize}

\textbf{Projets en Python}
\begin{itemize}
\item Desktop Cleaner : script pour automatiser le nettoyage et l'organisation des fichiers sur un ordinateur
\item Web Scraper : outil pour extraire des données à partir de sites web
\item Budget Tracker : application pour suivre et gérer un budget personnel
\item Compétences développées : programmation en Python, gestion de fichiers, extraction de données web, manipulation de données
\end{itemize}

\textbf{Projets en C}
\begin{itemize}
\item Système de gestion de bibliothèque : application pour gérer les livres, les membres et les emprunts d'une bibliothèque
\item Système de gestion bancaire : application simulant les opérations bancaires de base (dépôts, retraits, virements)
\item Calculatrice matricielle : programme pour effectuer des opérations mathématiques sur des matrices
\item Application de lecture de livres électroniques : logiciel pour lire et gérer des e-books
\item Guichet automatique bancaire (ATM) : simulation d'un GAB avec fonctionnalités de base
\item Compétences développées : programmation en C, gestion de données, algorithmique, interfaces utilisateur
\end{itemize}

Il est important de noter que les projets présentés ici ne représentent qu'une sélection de mes réalisations. Tous mes projets n'ont pas nécessairement abouti à un succès, mais chacun d'entre eux a été une source d'apprentissage précieuse, me permettant de développer mes compétences et d'apprendre de mes erreurs.\\
De plus, pour des raisons de confidentialité et de respect de la vie privée, certains projets, notamment ceux réalisés dans un cadre professionnel ou en collaboration avec des tiers, ne sont pas détaillés ou rendus publics dans ce CV.
\section{CENTRES D'INTÉRÊT ET ACTIVITÉS}
\begin{itemize}
\item Sports : pratique régulière de diverses activités sportives
\item Philosophie : lecture d'ouvrages et participation à des discussions et débats
\item Art, dessin et animation : pratique du dessin, de l'illustration et exploration de l'animation
\item Cuisine : expérimentation de différentes techniques et cuisines du monde
\item Cinéma : analyse de films et participation à des discussions sur le 7ème art
\end{itemize}
        
\end{document}