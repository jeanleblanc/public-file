\chapter{Annexe : Exercices}

\section*{Exercices chapitre 1}

\qs{Trying error}{
\begin{enumerate}
\item In a print statement, what happens if you leave out one of the parentheses, or both?
\item If you are trying to print a string, what happens if you leave out one of the quotation marks,
or both?
\item You can use a minus sign to make a negative number like -2. What happens if you put a plus
sign before a number? What about 2++2?
\item In math notation, leading zeros are ok, as in 09. What happens if you try this in Python?
What about 011?
\item What happens if you have two values with no operator between them?
\end{enumerate}
}
\sol 
\begin{enumerate}
\item Leaving out one or both parentheses in a \texttt{print} statement will result in a \texttt{SyntaxError}.
\item Omitting one or both quotation marks around a string will cause a \texttt{SyntaxError}.
\item Using a plus sign before a number has no effect (e.g., \texttt{+2} is the same as \texttt{2}). \texttt{2++2} raises a \texttt{SyntaxError} because the double plus is invalid syntax.
\item Leading zeros in integer literals are not allowed in Python 3. Both \texttt{09} and \texttt{011} will raise a \texttt{SyntaxError}.
\item Having two values with no operator between them (like \texttt{2 2}) will result in a \texttt{SyntaxError}.
\end{enumerate}

\qs{Python as a calculator}{
\begin{enumerate}
\item How many seconds are there in 42 minutes 42 seconds?
\item How many miles are there in 10 kilometers? Hint: there are 1.61 kilometers in a mile.
\item If you run a 10 kilometer race in 42 minutes 42 seconds, what is your average pace (time per
mile in minutes and seconds)? What is your average speed in miles per hour
\end{enumerate}
}

\sol \begin{verbatim}
#1 
print(42*60 + 42)
#2 
print(1/1.6*10)
#3
print("time per miles in second :", (42*60 + 42)/10, "in minutes : " (42 + 42/60)/10, "average speed : "10/((42+42/60)/60))
\end{verbatim}

\section*{Exercices chapitre 2}
\qs{Trying error}{
\begin{enumerate}
\item We’ve seen that n = 42 is legal. What about 42 = n?
\item How about x = y = 1?
\item In some languages every statement ends with a semi-colon, ;. What happens if you put a semi-colon at the end of a Python statement?
\item What if you put a period at the end of a statement?
\item In math notation you can multiply x and y like this: xy. What happens if you try that in
\end{enumerate}
}

\sol 
\begin{enumerate}
    \item \texttt{42 = n} is illegal in Python. It raises a \texttt{SyntaxError} because the left side of an assignment must be a variable name, not a literal value.
    \item \texttt{x = y = 1} is legal in Python. It assigns the value \texttt{1} to both \texttt{x} and \texttt{y} variables.
    \item Putting a semicolon (\texttt{;}) at the end of a Python statement is allowed but not required. It has no effect on the statement's execution.
    \item Placing a period (\texttt{.}) at the end of a statement will cause a \texttt{SyntaxError} because it is not valid Python syntax.
    \item Trying to multiply variables like \texttt{xy} in Python will raise a \texttt{NameError} because Python interprets it as a single variable name, not as multiplication. Use \texttt{x * y} instead.
\end{enumerate}

\qs{Python as a Calculator (script)}{
\begin{enumerate}
\item The volume of a sphere with radius \(r\)  is \(\frac{4}{3}\pi r^3\) . What is the volume of a sphere with radius 5?
\item Suppose the cover price of a book is 24.95 dollars, but bookstores get a 40\% discount. Shipping costs
3 dollars for the first copy and 75 cents for each additional copy. What is the total wholesale cost for
60 copies?
\item If I leave my house at 6:52 am and run 1 mile at an easy pace (8:15 per mile), then 3 miles at
tempo (7:12 per mile) and 1 mile at easy pace again, what time do I get home for breakfast?
\end{enumerate}
}
\sol
\begin{verbatim}
#1
r = 5
volume = 4/3 * 3.14 * r**3
print(volume)

#2
cp = 24.95
dis = 40/100
sc_1 = 3
sc_2 = 0.75
n = 60

total_cost = n*cp*dis + sc_1 + (n-1)*sc_2

print(total_cost)

#3
leave = 6*60 + 52
d1 = 2
d2 = 3
v1 = 8 + 15/60
v2 = 7 + 12/60

eta = (leave + d1*v1 + d2*v2)/60

print(eta)
\end{verbatim}

\section*{Exercices chapitre 3}

\qs{}{
Write a function named \texttt{right\_justify} that takes a string named s as a parameter
and prints the string with enough leading spaces so that the last letter of the string is in column 70
of the display.\\
Hint: Use string concatenation and repetition. Also, Python provides a built-in function called len
that returns the length of a string, so the value of \texttt{len('monty')} is 5.
}

\begin{verbatim}
>>> right_justify('monty')
                                monty
\end{verbatim}

\sol

\begin{verbatim}
def right_justify(s):
    print(" "*70+s)
    
right_justify("test")__
\end{verbatim}

\qs{}{
A function object is a value you can assign to a variable or pass as an 
argument. For
example, \texttt{do\_twice} is a function that takes a function object as an 
argument and calls it twice.\\
\begin{enumerate}
\item Type this example into a script and test it.
\item Modify \texttt{do\_twice} so that it takes two arguments, a function object and a value, and calls the
function twice, passing the value as an argument.
\item Copy the definition of \texttt{print\_twice} from earlier in this chapter to your script.
\item Use the modified version of \texttt{do\_twice} to call \texttt{print\_twice} twice, passing 'spam' as an
argument.
\item Define a new function called \texttt{do\_four} that takes a function object and a value and calls the
function four times, passing the value as a parameter. There should be only two statements in
the body of this function, not four.
\end{enumerate}
}

\begin{verbatim}
def do_twice(f):
    f()
    f()

# use example
def print_spam():
    print("spam")

do_twice(print_spam)
\end{verbatim}

\sol
\begin{enumerate}
    \item 
\begin{verbatim}
    > spam
      spam
\end{verbatim} 
    \item
\begin{verbatim}
def do_twice(f, text):
    f(text)
    f(text)

def print_spam(text):
    print(text)

do_twice(print_spam, "try")
\end{verbatim}
    \item voir 4.
    \item 
\begin{verbatim}
def do_twice(f, text):
    f(text)
    f(text)

def print_twice(text):
    print(text)
    print(text)

do_twice(print_twice, "tentative") 
\end{verbatim}
    \item
\begin{verbatim}
def do_two(f, val):
    f(val)
    f(val)
 
def do_four(f, val):
    do_two(f, val)
    do_two(f, val)

do_four(print, "valeur")
\end{verbatim}
\end{enumerate}

\qs{}{
    \nt{
    This exercise should be done using only the statements and other features we
have learned so far.
    }
    
\begin{enumerate}
    \item Write a function that draws a grid like the following.\\
    \item Write a function that draws a similar grid with four rows and four columns.
   
\end{enumerate}
}

\begin{verbatim}
# for point 1 : 
+ - - - - + - - - - +
|         |         |
|         |         |
|         |         |
|         |         |
+ - - - - + - - - - +
|         |         |
|         |         |
|         |         |
|         |         |
+ - - - - + - - - - +
\end{verbatim}

\sol
\begin{verbatim}
def line(n):
    print("+ - - - -"*n,"+")

def space(n):
    print("|        "*n, "|")
    print("|        "*n, "|")
    print("|        "*n, "|")

def print_grid(n):
    line(n), space(n)
    line(n), space(n)
    line(n), space(n),line(n)
    
print_grid(3)
\end{verbatim}


\section*{Exo 4}
\subsection*{4.3}
\qs{}{
Write a function called square that takes a parameter named t, which is a turtle. It
should use the turtle to draw a square.\\
Write a function call that passes bob as an argument to square, and then run the
program again.
}
\sol 
\begin{verbatim}
import turtle 

boss = turtle.Turtle()

def square(t):
    for i in range(4):
        t.fd(100)
        t.rt(90)
        
        
square(boss)
\end{verbatim}

\qs{}{
Add another parameter, named length, to square. Modify the body so length of the
sides is length, and then modify the function call to provide a second argument. Run
the program again. Test your program with a range of values for length.
}

\sol 
\begin{verbatim}
import turtle 

boss = turtle.Turtle()

def square(t, length):
    for i in range(4):
        t.fd(length)
        t.rt(90)
        
        
square(boss, 30)
\end{verbatim}

\qs{}{
 Make a copy of square and change the name to polygon. Add another parameter
named n and modify the body so it draws an n-sided regular polygon. Hint: The
exterior angles of an n-sided regular polygon are 360/n degrees.
}

\sol
\begin{verbatim}
import turtle 

boss = turtle.Turtle()

def polygone(t, length, n):
    for i in range(n):
        t.fd(length)
        t.rt(360/n)
        
        
polygone(boss, 30, 5)
\end{verbatim}

\qs{}{
Write a function called circle that takes a turtle, t, and radius, r, as parameters and
that draws an approximate circle by calling polygon with an appropriate length and
number of sides. Test your function with a range of values of r.
Hint: figure out the circumference of the circle and make sure that length * n =
circumference.
}
\sol
\begin{verbatim}
import turtle
import math

boss = turtle.Turtle()

def polygone(t, length, n):
    for i in range(n):
        t.fd(length)
        t.rt(360/n)
        
def circle(t,  r):
   polygone(t, 2 * math.pi*r / 300, 300) 
   
circle(boss, 30)
\end{verbatim}
\qs{}{
Make a more general version of circle called arc that takes an additional parameter
angle, which determines what fraction of a circle to draw. angle is in units of degrees,
so when angle=360, arc should draw a complete circle.
}

\sol 
\begin{verbatim}
import turtle
import math

boss = turtle.Turtle()

def polygone(t, length, n, angle):
    for i in range(int(n/(360/angle))):
        t.fd(length)
        t.rt(360/n)
        
def arc(t,  r, angle):
   polygone(t, 2*math.pi*r/300, 300, angle)
   
arc(boss, 30, 90)
\end{verbatim}
\subsection*{4.12}


\qs{}{
\begin{enumerate}
    \item Draw a stack diagram that shows the state of the program while executing circle(bob,
radius). You can do the arithmetic by hand or add print statements to the code.
    \item The version of arc in Section 4.7 is not very accurate because the linear approximation of the
circle is always outside the true circle. As a result, the Turtle ends up a few pixels away from
the correct destination. My solution shows a way to reduce the effect of this error. Read the
code and see if it makes sense to you. If you draw a diagram, you might see how it works
\end{enumerate}}
\sol  trivial

\qs{}{
    Write an appropriately general set of functions that can draw flowers as in Figure 4.1.
}
\sol 
\begin{verbatim}
    import math
    import arc
    import turtle
    
    def petal(t, r, angle):
        for i in range(2):
            arc(t, r, angle)
            t.rt(180-angle)
    
    def flower(t, n, r, angle):
        for i in range(n):
            petal(t, r, angle)
            t.rt(360.0/n)
    
    def move(t, lenght)
        t.pu()
        t.fd(lenght)
        t.pd()
\end{verbatim}
\qs{}{
Write an appropriately general set of functions that can draw shapes as in Figure 4.2.
} 
\sol 
\begin{verbatim}
import math
import turtle
t = turtle.Turtle()

def triangle(t, r, angle):
    y = r*math.sin(angle*math.pi/180)

    t.rt(angle)
    t.fd(r)
    t.lt(angle + 90)
    t.(2*y)
    t.(angle + 90)
    t.fd(r)
    t.lt(180- angle)

def polygone(t, r, n):
    angle = 360.0/n

    for i in range(n):
        triangle(t, r, angle/2)
        t.lt(angle)

polygone(t, 100, 10)
\end{verbatim}

\qs{}{
The letters of the alphabet can be constructed from a moderate number of basic ele-
ments, like vertical and horizontal lines and a few curves. Design an alphabet that can be drawn
with a minimal number of basic elements and then write functions that draw the letters
}
\sol
\begin{verbatim}
    import math
import turtle
t= turtle.Turtle()
t.hideturtle()

   

# turtle point vers droite initialement 
def slash(t):
    t.lt(70)
    t.fd(100)
    t.lt(180)
    t.fd(100)
    t.lt(110)

def back_slash(t):
    t.lt(110)
    t.fd(100)
    t.lt(180)
    t.fd(100)
    t.lt(70)

def up_hline(t):
   t.lt(90) 
   t.pu()
   t.fd(100)
   t.pd()
   t.rt(90)
   t.fd(100)
   t.lt(180)
   t.fd(100)
   t.lt(90)
   t.pu()
   t.fd(100)
   t.pd()
   t.lt(90)

def down_hline(t):
   t.fd(100)
   t.lt(180)
   t.fd(100)
   t.lt(180)

def mid_hline(t):
   t.lt(90) 
   t.pu()
   t.fd(50)
   t.pd()
   t.rt(90)
   t.fd(100)
   t.bk(100)
   t.lt(90)
   t.pu()
   t.fd(50)
   t.pd()
   t.lt(90)

def hline(t):
   t.lt(90)
   t.fd(100)
   t.bk(100)
   t.rt(90)
def A(t):
   slash(t)
   t.pu()
   t.fd(72)
   t.pd()
   back_slash(t)
   t.pu()
   t.bk(72)
   t.pd()
   mid_hline(t)

def E(t):

A(t)

turtle.done()
\end{verbatim}