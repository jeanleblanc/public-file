\documentclass{report}

\input{preamble}
\input{macros}
\input{letterfonts}

\title{\Huge{Programmation }\\Python}
\author{\huge{Jean A. Aboutboul}}
\date{Spring 2024}

\begin{document}

\maketitle
\newpage% or \cleardoublepage
% \pdfbookmark[<level>]{<title>}{<dest>}
\pdfbookmark[section]{\contentsname}{toc}
\tableofcontents
\pagebreak
\chapter{The Way of the Program}

The most important skill of a computer scientist is \textbf{problem solving}, which is the ability to formulate problems, think creatively about solutions, and express a solution clearly and accurately.

\section{What is a Program?}

\define{A program} is a sequence of instructions that specifies how to perform a computation (mathematical or symbolic). Details differ depending on the language used, but basic instructions are common in every language:

\begin{itemize}
\item \textbf{input} Get data from outside.
\item \textbf{output} Display data, save it, send it, etc.
\item \textbf{math} Perform basic mathematical operations.
\item \textbf{conditional execution} Check conditions and run the appropriate code.
\item \textbf{repetition} Perform an action repeatedly, with variations.
\end{itemize}

\section{Running Python}

Install it:

\begin{verbatim}
sudo apt install python3.8 python3-pip
sudo pacman -S python
...
python --version # Check the installation
\end{verbatim}

The \textbf{Python interpreter} is a program that reads and executes Python code.

\begin{verbatim}
python   # In bash
python3  # In macOS
\end{verbatim}

\section{First Program}

\begin{verbatim}
print("Hello, World!") # Display Hello, World!
\end{verbatim}

The \textbf{print function} displays a result on the screen. The quotation marks do not appear in the result. The parentheses indicate that "print" is a function (see Chapter 3).

\section{Arithmetic Operators}

\define{Operators} are provided by Python and represent computations:

\begin{itemize}
\item +
\item -
\item $\ast$
\item /
\item $\ast\ast$ (exponential; in Python, it is not "\textasciicircum", which is the bitwise XOR operator)
\end{itemize}

\section{Values and Types}

\define{A value} is a basic thing that a program works with. Values belong to different \textbf{types}:

\begin{itemize}
\item \textbf{integer} 2, 3, etc.
\item \textbf{floating-point number} 2.0, 3.42, etc.
\item \textbf{string} Characters (letters are strung together)
\end{itemize}

You can ask the computer what type something is:

\begin{verbatim}
type(2)              # int
type(2.0)            # float
type("hello world")  # str
type("3")            # str (because of the quotation marks)
\end{verbatim}

\section{Formal vs. Natural Languages}

\define{Natural languages} are the languages people speak. They are not designed by people and have evolved naturally.\\
\define{Formal languages} are designed by people for specific applications (math, chemistry, etc.). \textbf{Programming languages} are formal languages that have been designed to express computations.\\
Formal languages tend to have strict \textbf{syntax} that comes in two ways: \textbf{tokens} and \textbf{structure}.\\
\define{Tokens} are the basic elements of the language. There are legal and non-legal tokens (e.g., \$ is not a legal token in math).\\
\define{Rules} pertain to the way tokens are combined. There are also legal and illegal ways to arrange tokens (e.g., 3 + = / 5 uses all legal tokens but has an illegal arrangement).\\
\textbf{Parsing} is the process of figuring out the structure of a statement in a language.\\
Formal languages provide advantages: lack of ambiguity, redundancy, and literalness.\\

\section{Debugging}

Programmers make mistakes; those mistakes are called \textbf{bugs}, and the process of tracking them is called \textbf{debugging}.

\section{Exercises}

\textbf{Remember}: Test bugs to see what happens.





\chapter{Variables, expressions and statements}

\define{A variable} is a name taht refers to a value.
\section{Assignement statements}
An assignement statements create a new variables. In python it is very simple 
\begin{verbatim}
name = value

temps = 42
phrase = "hello world"
\end{verbatim}

\section{Variables names}
Can be as long as you want, can contain letter and num but  can't beggin
with a num, we avoid uppercase but its legal, underscore is ok. Also 
can't use python3 keywords which are : 
\begin{verbatim}
False     class      finally   is         return
None      continue   for       lambda     try
True      def        from      nonlocal   while
and       del        global    not        with
as        elif       if        or         yield
assert    else       import    pass
break     except     in        raise
\end{verbatim}

\section{Expressions and statements}
\define{An expression} is a combination of value, variables and operators.\\
When you type an expression at the prompt, the interpreter evaluates it.\\
\define{A statement} is a unit of code that has an effect.

\section{Script mode}
In the terminal, when you type python, it enter in 
\textbf{interactive mode}, which mean that we interact directly with the interpreter.\\
Script mode consist as saving code in a file called a \textbf{script} and run 
the interpreter in \textbf{script mode}.\\

Create a file with a \textit{.py} then run it in the terminal
by using the following : 
\begin{verbatim}
./file.py

#may need to make it executable 
sudo chmod +x file.py
\end{verbatim}

\textbf{Ordre of operations} : PEMDAS

\section{String operations}
We can't perform mathematical operations on strings, but there is two exeptions : \\
+ wich perform string concatenation (join strings end to end) and $\ast$ who 
perform repetition. 
\begin{verbatim}
one = "test"
two = "ok"
one + two = testok

one * 3 = testtesttest
\end{verbatim}

\section{Comments}
\define{Comments} are not that are not read by the interpretor.\\
In python it use the symbold \# and everything after will be a comment.

\section{Debugging}
Three kin of errors can occus in a program : 
\begin{itemize}
    \item \textbf{Syntax error} refers to structure and rules of a program.
    \item \textbf{Runtime error}, error that does not appear until after the program has
    started running. Called exceptions.
    \item \textbf{Semantic error} are related to the meaning, the program will 
    run but not do what is expect.
\end{itemize}
\section{Exercices}
nothing really intersting

\chapter{Functions}
\define{A function} (in the context of programming) is 
a named sequence of statement that performs a computation.\\
Define a fuction : specify the name and the sequence of statement.

\section{Function calls}
\define{The argument} is the expression in parentheses.
\begin{verbatim}
type(42) > int
int(42.0) > 42
...
\end{verbatim}
\section{Math function}
\define{A module is a file that contains a collection of related functions}.\\
Python provide a math module.\\
Before we can use a module we need to \textbf{import} it. Then
to acces the functions you have to specify the name of the module
and the name of the function separated by a dot.
\begin{verbatim}
import math

radian = 2 * math.pi
height = math.sin(radian)
testvalue = math.log10(3) #default is e

test2 = math.sqrt(2)
\end{verbatim}

\section{Adding new functions}
\textbf{A functions definition} specifies the name of a new function
and the sequence of statement that run when the function is called.\\

def is a keyword that indicates that this is a function def. Emtpy 
parenthesis indicate that the function does not take any argument. The 
first line is the \textbf{header} and has to finish with a parenthesis. The 
rest is called the \textbf{body}. To end the function you have to enter an empty line.

\begin{verbatim}
def print_lyrics():         #header
    print("I'm cold")       #body
    print("but its ok")
                            #empty line
\end{verbatim}
Defining a function creates a function object, which has type function.\\
The syntax for calling a created function is the same as for built-in function.\\

If you define a function, the definition will be read as any other statement 
in the programme, but the effect is to create function objects. The body, in the 
oder hand, will not be read (and execute) until the function is called.\\
\define{The flow of execution} is the oreder statement run in.
Execution allways begins at the first statement of the program and run every 
statement, one at a time, in order from top to bottom. But as a function
pass few statement (the body) but can go back to it later in the program.

\section{Parametres and arguments}
Some function require one ore more argument. Inside the function
the argument are assigned to variables called \textbf{parameters}.

\begin{verbatim}
def the_function(the parameter)
    print(parameter)


the_function("test")
> test
\end{verbatim}\\

When you create a variable inside a function it is said to be \textbf{local}
which mean that it only exists inside the function.

\define{A stack diagram} is a diagram (as a state diagrams) that show the value of each variable,
but also the function each variable belongs to. Each function is represented by a frame\\
\define{A frame} is a box with the name of a cuntion beside it and the parameters
and variables of the function inside it. Frames are arranged in stack taht indicates which function
called which and so on. \texttt{\_\_main\_\_} is the name for the topmost frame. Each parameter
refer to the same value as its corresponding argument.\\
The functions \textbf{\texttt{traceback()}} tells you what program file 
the error  occured in and what line, and what function were executing at the time. Also 
show the line of code taht caused the error.

\section{Fruitful functions and void function}
\define{"Fruitful functions"} (arbitrary name) are functions that return results.\\
\define{void functions} are functions that perform an action but dont return a value.

\begin{Example}{}{}
\begin{verbatim}
def print_twice(value)
    print(value)
    print(value)

result = print_twice('bing')
print('result')
> None #(as the function print_twice is a void function)
\end{verbatim}
\end{Example}

We use for :
\begin{itemize}
    \item Creating a group of statements
    \item Make programme smaller
    \item Dividing long program into smaller part
    \item Useful for many programs
\end{itemize}


\chapter{Case Study : Interface disign}

\section{The Turtle Module}

\begin{verbatim}
#check if the module is install : 
import turtle
bob = turtle.Turtle()

#if it's not 
sudo apt install python-tk
\end{verbatim}

\begin{verbatim}
import turtle
bob = turtle.Turtle #create a function name Turtle that create an object and asign it to a variable

print(bob)
>> turtle.Turtle object at 04NX53 #bob refer to an object with type Turtle
\end{verbatim}
We can also call a \textbf{Method} to move the object around the window. Its like
a function but use a bit defferent syntax.
\begin{verbatim}
bob.fd(100) #argument in pixels
bob.bk(100)
bob.lt(90)  #argument is an angle
bob.rt(100)
\end{verbatim}

\textbf{For statement} : loop, ends with a comma and take 
in the body any number of statements.
\begin{verbatim}
for i in rang(4):
    bob.fd(100)
    bob.lt(90)
\end{verbatim}

\define{Encapsulation} is the process of putting a program into a function.\\
\define{Generalization} using variables instead of constant values.\\
\define{Interface of a function} is a summary of how it is used.\\
\define{Refactoring} is the process of rearengin a program to improve interfaces 
and faciliatete code re-use.\\

So we can create a "dev plan" based on the previous definitions : 
\begin{enumerate}
    \item writing small programme (without functions)
    \item Identify coherent pieces and encapsulate it.
    \item Generalize the new fucntions
    \item Repeat 1-3 until you have a set of working functions. then copy-paste
    \item refactoring
\end{enumerate}

\define{Docstring} is a string at the begging of a function that explain the interface. They are triple quotes
\begin{verbatim}
def arc(t, n, length, angle):
    """ draw n lines segments
    for having an idead
    this is a dot string that is in multiple lines"""
rest of the code

\end{verbatim}
\textbf{Pre conditions} : true before the functions start executing.\\
\textbf{Post conditions} in the other hand, include the intend effect of the functions.\\
Then if the bug is in the pre-conditions, the problem is from the caller and not the functions, but
if the pre-conditions are satisfied, but not post-conditions, the function is the problem.




\chapter{Annexe : Exercices}

\section*{Exercices chapitre 1}

\qs{Trying error}{
\begin{enumerate}
\item In a print statement, what happens if you leave out one of the parentheses, or both?
\item If you are trying to print a string, what happens if you leave out one of the quotation marks,
or both?
\item You can use a minus sign to make a negative number like -2. What happens if you put a plus
sign before a number? What about 2++2?
\item In math notation, leading zeros are ok, as in 09. What happens if you try this in Python?
What about 011?
\item What happens if you have two values with no operator between them?
\end{enumerate}
}
\sol 
\begin{enumerate}
\item Leaving out one or both parentheses in a \texttt{print} statement will result in a \texttt{SyntaxError}.
\item Omitting one or both quotation marks around a string will cause a \texttt{SyntaxError}.
\item Using a plus sign before a number has no effect (e.g., \texttt{+2} is the same as \texttt{2}). \texttt{2++2} raises a \texttt{SyntaxError} because the double plus is invalid syntax.
\item Leading zeros in integer literals are not allowed in Python 3. Both \texttt{09} and \texttt{011} will raise a \texttt{SyntaxError}.
\item Having two values with no operator between them (like \texttt{2 2}) will result in a \texttt{SyntaxError}.
\end{enumerate}

\qs{Python as a calculator}{
\begin{enumerate}
\item How many seconds are there in 42 minutes 42 seconds?
\item How many miles are there in 10 kilometers? Hint: there are 1.61 kilometers in a mile.
\item If you run a 10 kilometer race in 42 minutes 42 seconds, what is your average pace (time per
mile in minutes and seconds)? What is your average speed in miles per hour
\end{enumerate}
}

\sol \begin{verbatim}
#1 
print(42*60 + 42)
#2 
print(1/1.6*10)
#3
print("time per miles in second :", (42*60 + 42)/10, "in minutes : " (42 + 42/60)/10, "average speed : "10/((42+42/60)/60))
\end{verbatim}

\section*{Exercices chapitre 2}
\qs{Trying error}{
\begin{enumerate}
\item We’ve seen that n = 42 is legal. What about 42 = n?
\item How about x = y = 1?
\item In some languages every statement ends with a semi-colon, ;. What happens if you put a semi-colon at the end of a Python statement?
\item What if you put a period at the end of a statement?
\item In math notation you can multiply x and y like this: xy. What happens if you try that in
\end{enumerate}
}

\sol 
\begin{enumerate}
    \item \texttt{42 = n} is illegal in Python. It raises a \texttt{SyntaxError} because the left side of an assignment must be a variable name, not a literal value.
    \item \texttt{x = y = 1} is legal in Python. It assigns the value \texttt{1} to both \texttt{x} and \texttt{y} variables.
    \item Putting a semicolon (\texttt{;}) at the end of a Python statement is allowed but not required. It has no effect on the statement's execution.
    \item Placing a period (\texttt{.}) at the end of a statement will cause a \texttt{SyntaxError} because it is not valid Python syntax.
    \item Trying to multiply variables like \texttt{xy} in Python will raise a \texttt{NameError} because Python interprets it as a single variable name, not as multiplication. Use \texttt{x * y} instead.
\end{enumerate}

\qs{Python as a Calculator (script)}{
\begin{enumerate}
\item The volume of a sphere with radius \(r\)  is \(\frac{4}{3}\pi r^3\) . What is the volume of a sphere with radius 5?
\item Suppose the cover price of a book is 24.95 dollars, but bookstores get a 40\% discount. Shipping costs
3 dollars for the first copy and 75 cents for each additional copy. What is the total wholesale cost for
60 copies?
\item If I leave my house at 6:52 am and run 1 mile at an easy pace (8:15 per mile), then 3 miles at
tempo (7:12 per mile) and 1 mile at easy pace again, what time do I get home for breakfast?
\end{enumerate}
}
\sol
\begin{verbatim}
#1
r = 5
volume = 4/3 * 3.14 * r**3
print(volume)

#2
cp = 24.95
dis = 40/100
sc_1 = 3
sc_2 = 0.75
n = 60

total_cost = n*cp*dis + sc_1 + (n-1)*sc_2

print(total_cost)

#3
leave = 6*60 + 52
d1 = 2
d2 = 3
v1 = 8 + 15/60
v2 = 7 + 12/60

eta = (leave + d1*v1 + d2*v2)/60

print(eta)
\end{verbatim}

\section*{Exercices chapitre 3}

\qs{}{
Write a function named \texttt{right\_justify} that takes a string named s as a parameter
and prints the string with enough leading spaces so that the last letter of the string is in column 70
of the display.\\
Hint: Use string concatenation and repetition. Also, Python provides a built-in function called len
that returns the length of a string, so the value of \texttt{len('monty')} is 5.
}

\begin{verbatim}
>>> right_justify('monty')
                                monty
\end{verbatim}

\sol

\begin{verbatim}
def right_justify(s):
    print(" "*70+s)
    
right_justify("test")__
\end{verbatim}

\qs{}{
A function object is a value you can assign to a variable or pass as an 
argument. For
example, \texttt{do\_twice} is a function that takes a function object as an 
argument and calls it twice.\\
\begin{enumerate}
\item Type this example into a script and test it.
\item Modify \texttt{do\_twice} so that it takes two arguments, a function object and a value, and calls the
function twice, passing the value as an argument.
\item Copy the definition of \texttt{print\_twice} from earlier in this chapter to your script.
\item Use the modified version of \texttt{do\_twice} to call \texttt{print\_twice} twice, passing 'spam' as an
argument.
\item Define a new function called \texttt{do\_four} that takes a function object and a value and calls the
function four times, passing the value as a parameter. There should be only two statements in
the body of this function, not four.
\end{enumerate}
}

\begin{verbatim}
def do_twice(f):
    f()
    f()

# use example
def print_spam():
    print("spam")

do_twice(print_spam)
\end{verbatim}

\sol
\begin{enumerate}
    \item 
\begin{verbatim}
    > spam
      spam
\end{verbatim} 
    \item
\begin{verbatim}
def do_twice(f, text):
    f(text)
    f(text)

def print_spam(text):
    print(text)

do_twice(print_spam, "try")
\end{verbatim}
    \item voir 4.
    \item 
\begin{verbatim}
def do_twice(f, text):
    f(text)
    f(text)

def print_twice(text):
    print(text)
    print(text)

do_twice(print_twice, "tentative") 
\end{verbatim}
    \item
\begin{verbatim}
def do_two(f, val):
    f(val)
    f(val)
 
def do_four(f, val):
    do_two(f, val)
    do_two(f, val)

do_four(print, "valeur")
\end{verbatim}
\end{enumerate}

\qs{}{
    \nt{
    This exercise should be done using only the statements and other features we
have learned so far.
    }
    
\begin{enumerate}
    \item Write a function that draws a grid like the following.\\
    \item Write a function that draws a similar grid with four rows and four columns.
   
\end{enumerate}
}

\begin{verbatim}
# for point 1 : 
+ - - - - + - - - - +
|         |         |
|         |         |
|         |         |
|         |         |
+ - - - - + - - - - +
|         |         |
|         |         |
|         |         |
|         |         |
+ - - - - + - - - - +
\end{verbatim}

\sol
\begin{verbatim}
def line(n):
    print("+ - - - -"*n,"+")

def space(n):
    print("|        "*n, "|")
    print("|        "*n, "|")
    print("|        "*n, "|")

def print_grid(n):
    line(n), space(n)
    line(n), space(n)
    line(n), space(n),line(n)
    
print_grid(3)
\end{verbatim}


\section*{Exo 4}
\subsection*{4.3}
\qs{}{
Write a function called square that takes a parameter named t, which is a turtle. It
should use the turtle to draw a square.\\
Write a function call that passes bob as an argument to square, and then run the
program again.
}
\sol 
\begin{verbatim}
import turtle 

boss = turtle.Turtle()

def square(t):
    for i in range(4):
        t.fd(100)
        t.rt(90)
        
        
square(boss)
\end{verbatim}

\qs{}{
Add another parameter, named length, to square. Modify the body so length of the
sides is length, and then modify the function call to provide a second argument. Run
the program again. Test your program with a range of values for length.
}

\sol 
\begin{verbatim}
import turtle 

boss = turtle.Turtle()

def square(t, length):
    for i in range(4):
        t.fd(length)
        t.rt(90)
        
        
square(boss, 30)
\end{verbatim}

\qs{}{
 Make a copy of square and change the name to polygon. Add another parameter
named n and modify the body so it draws an n-sided regular polygon. Hint: The
exterior angles of an n-sided regular polygon are 360/n degrees.
}

\sol
\begin{verbatim}
import turtle 

boss = turtle.Turtle()

def polygone(t, length, n):
    for i in range(n):
        t.fd(length)
        t.rt(360/n)
        
        
polygone(boss, 30, 5)
\end{verbatim}

\qs{}{
Write a function called circle that takes a turtle, t, and radius, r, as parameters and
that draws an approximate circle by calling polygon with an appropriate length and
number of sides. Test your function with a range of values of r.
Hint: figure out the circumference of the circle and make sure that length * n =
circumference.
}
\sol
\begin{verbatim}
import turtle
import math

boss = turtle.Turtle()

def polygone(t, length, n):
    for i in range(n):
        t.fd(length)
        t.rt(360/n)
        
def circle(t,  r):
   polygone(t, 2 * math.pi*r / 300, 300) 
   
circle(boss, 30)
\end{verbatim}
\qs{}{
Make a more general version of circle called arc that takes an additional parameter
angle, which determines what fraction of a circle to draw. angle is in units of degrees,
so when angle=360, arc should draw a complete circle.
}

\sol 
\begin{verbatim}
import turtle
import math

boss = turtle.Turtle()

def polygone(t, length, n, angle):
    for i in range(int(n/(360/angle))):
        t.fd(length)
        t.rt(360/n)
        
def arc(t,  r, angle):
   polygone(t, 2*math.pi*r/300, 300, angle)
   
arc(boss, 30, 90)
\end{verbatim}
\subsection*{4.12}


\qs{}{
\begin{enumerate}
    \item Draw a stack diagram that shows the state of the program while executing circle(bob,
radius). You can do the arithmetic by hand or add print statements to the code.
    \item The version of arc in Section 4.7 is not very accurate because the linear approximation of the
circle is always outside the true circle. As a result, the Turtle ends up a few pixels away from
the correct destination. My solution shows a way to reduce the effect of this error. Read the
code and see if it makes sense to you. If you draw a diagram, you might see how it works
\end{enumerate}}
\sol  trivial

\qs{}{
    Write an appropriately general set of functions that can draw flowers as in Figure 4.1.
}
\sol 
\begin{verbatim}
    import math
    import arc
    import turtle
    
    def petal(t, r, angle):
        for i in range(2):
            arc(t, r, angle)
            t.rt(180-angle)
    
    def flower(t, n, r, angle):
        for i in range(n):
            petal(t, r, angle)
            t.rt(360.0/n)
    
    def move(t, lenght)
        t.pu()
        t.fd(lenght)
        t.pd()
\end{verbatim}
\qs{}{
Write an appropriately general set of functions that can draw shapes as in Figure 4.2.
} 
\sol 
\begin{verbatim}
import math
import turtle
t = turtle.Turtle()

def triangle(t, r, angle):
    y = r*math.sin(angle*math.pi/180)

    t.rt(angle)
    t.fd(r)
    t.lt(angle + 90)
    t.(2*y)
    t.(angle + 90)
    t.fd(r)
    t.lt(180- angle)

def polygone(t, r, n):
    angle = 360.0/n

    for i in range(n):
        triangle(t, r, angle/2)
        t.lt(angle)

polygone(t, 100, 10)
\end{verbatim}

\qs{}{
The letters of the alphabet can be constructed from a moderate number of basic ele-
ments, like vertical and horizontal lines and a few curves. Design an alphabet that can be drawn
with a minimal number of basic elements and then write functions that draw the letters
}
\sol
\begin{verbatim}
    import math
import turtle
t= turtle.Turtle()
t.hideturtle()

   

# turtle point vers droite initialement 
def slash(t):
    t.lt(70)
    t.fd(100)
    t.lt(180)
    t.fd(100)
    t.lt(110)

def back_slash(t):
    t.lt(110)
    t.fd(100)
    t.lt(180)
    t.fd(100)
    t.lt(70)

def up_hline(t):
   t.lt(90) 
   t.pu()
   t.fd(100)
   t.pd()
   t.rt(90)
   t.fd(100)
   t.lt(180)
   t.fd(100)
   t.lt(90)
   t.pu()
   t.fd(100)
   t.pd()
   t.lt(90)

def down_hline(t):
   t.fd(100)
   t.lt(180)
   t.fd(100)
   t.lt(180)

def mid_hline(t):
   t.lt(90) 
   t.pu()
   t.fd(50)
   t.pd()
   t.rt(90)
   t.fd(100)
   t.bk(100)
   t.lt(90)
   t.pu()
   t.fd(50)
   t.pd()
   t.lt(90)

def hline(t):
   t.lt(90)
   t.fd(100)
   t.bk(100)
   t.rt(90)
def A(t):
   slash(t)
   t.pu()
   t.fd(72)
   t.pd()
   back_slash(t)
   t.pu()
   t.bk(72)
   t.pd()
   mid_hline(t)

def E(t):

A(t)

turtle.done()
\end{verbatim}
\end{document}
